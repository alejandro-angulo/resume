\documentclass{article}

\usepackage{array}
\usepackage{enumitem}
\usepackage[letterpaper, portrait, top=0.5in, right=0.5in, left=0.5in, bottom=1in]{geometry}
\usepackage{hyperref}
\usepackage{multirow}
\usepackage{titlesec}
\usepackage{comment}
\usepackage[usenames, dvipsnames]{xcolor}
\usepackage{fancyhdr}
\usepackage{fontspec}
\setmonofont{Hack Nerd Font}

\definecolor{blue}{HTML}{2079C7}
\definecolor{gray}{HTML}{666666}

\newcolumntype{L}[1]{>{\raggedright\let\newline\\\arraybackslash\hspace{0pt}}m{#1}}
\newcolumntype{C}[1]{>{\centering\let\newline\\\arraybackslash\hspace{0pt}}m{#1}}
\newcolumntype{R}[1]{>{\raggedleft\let\newline\\\arraybackslash\hspace{0pt}}m{#1}}

\pagenumbering{gobble}

\renewcommand{\familydefault}{\sfdefault}
\renewcommand{\headrulewidth}{0pt}
\renewcommand{\footrulewidth}{0.4pt}

\setlist[itemize]{noitemsep}

\titleformat{\section}
    {\color{blue}\normalfont\Large\bfseries}
    {\color{blue}\thesection}{1em}{\large\uppercase}
\titleformat{\subsection}
    {\normalfont\Large\bfseries}
    {\thesection}{1em}{\normalsize}
\titleformat{\subsubsection}
    {\color{gray}\normalfont\Large\bfseries}
    {\color{gray}\thesection}{1em}{\small}

\titlespacing*{\section}
    {0em}{0em}{0em}
\titlespacing*{\subsection}
    {0em}{1em}{0em}
\titlespacing*{\subsubsection}
    {0em}{0em}{0em}

\begin{document}

\noindent

\pagestyle{fancy}
\fancyfoot[L]{
    \small{The source code for this document is available at \href{https://github.com/alejandro-angulo/resume}{https://github.com/alejandro-angulo/resume}}\\
    \small{If you have \href{https://nixos.wiki/wiki/Nix_package_manager}{nix} installed you can run the command below to generate the latest version of this document}\\
    \scriptsize{\texttt{env EMAIL='\email' PHONENUMBER='\phonenumber' nix run --refresh --experimental-features 'nix-command flakes' github:alejandro-angulo/resume/main}}
}

\begin{minipage}[t]{\textwidth}
\begin{tabular}{L{.4\textwidth} l l}
    \multirow{2}{*}{\Huge{ \textbf{Alejandro Angulo}}} &  \href{https://github.com/alejandro-angulo}{\small{\texttt{}} \space https://github.com/alejandro-angulo} & \href{https://alejandr0angul0.dev}{\small{\texttt{爵}} \space https://alejandr0angul0.dev}\\
    & \href{mailto:\email}{\small{\texttt{}} \space \email} & \small{\texttt{}} \space \phonenumber\\
    \hline
\end{tabular}
\vspace{1em}
\end{minipage}
\begin{minipage}[t]{.8\textwidth}
\section*{Professional Experience}
\subsection*{Sure  --- \textit{Senior Software Engineer}}
\subsubsection*{February 2020 - Present}
\parbox[t]{.5\textwidth}{\raggedright%
\begin{itemize}
\vspace{-0.5em}
    \item Develop and maintain django applications
    \item Help unstick teammates
    \item Leveraged pre-commit to help developers run linters and formatters automatically when they attempt to mkae a commit
    \item Configured CI workflows to push wheels of internal applications to our private package index
\end{itemize}}
\parbox[t]{.5\textwidth}{\raggedright%
\begin{itemize}
\vspace{-1.5em}
    \item Pushing for more/better documentation and introduced additional tooling to help (configured repos with sphinx and set up CI pipelines to generate OpenAPI schemas)
\end{itemize}}
\vspace{-1.5em}
\subsection*{Everbridge (formerly NC4) --- \textit{Software Engineer}}
\subsubsection*{October 2018 - February 2020}
\parbox[t]{.5\textwidth}{\raggedright%
\begin{itemize}
\vspace{-0.5em}
    \item Develop and maintain Python and PHP applications
    \item Migrate VCS from SVN to Git
    \item Re-architect Python application to improve modularity
    \item Introduce best practices (follow PEP8 for Python, begin linting code, etc.)
\end{itemize}}
\parbox[t]{.5\textwidth}{\raggedright%
\begin{itemize}
\vspace{-1.5em}
    \item Streamline code review process (integrate with source control)
    \item Maintain and develop a RESTful API written in Python capable of generating reports for end users
    \item Work with product managers to implement "behind the scenes" changes (upgrading language versions, databases, etc.)
\end{itemize}}
\vspace{-1.5em}
\subsection*{MedQIA --- \textit{Software Engineer}}
\subsubsection*{April 2017 - September 2018}
\parbox[t]{.5\textwidth}{\raggedright%
\begin{itemize}
\vspace{-1.5em}
    \item Develop and maintain a Java desktop application used by clinical readers and lab technologists to assist in clinical trials
    \item Document and validate systems to comply with FDA regulations
    \item Work on Python applications that keep track of the status of scans
    \item Maintain and develop a web client written in PHP and JavaScript that provides a dashboard for the data team
\end{itemize}}
\parbox[t]{.5\textwidth}{\raggedright%
\vspace{-1.5em}
\begin{itemize}
    \item Maintain and develop a RESTful API written in Python capable of generating reports for end users
    \item Write Python scripts that plug in to a Java application in order to provide additional functionality (dynamic loading of patients, automating certain tasks for end users, etc.)
    \item Automate build process for Java applications using Gradle
\end{itemize}}
\vspace{-1.5em}
\subsection*{UC Davis College of Engineering IT Shared Services --- \textit{Student Web Developer and Applications Programmer}}
\subsubsection*{May 2014 - April 2017}
\parbox[t]{.5\textwidth}{\raggedright%
\begin{itemize}
\vspace{-1.5em}
    \item Migrate ITS's existing websites to a new server
    \item Write PHP scripts to scrape data from third-party sites in order to visualize energy consumption
\end{itemize}}
\parbox[t]{.5\textwidth}{\raggedright%
\vspace{-1.5em}
\begin{itemize}
    \item Ensure compliance with security standards
    \item Design responsive layouts using CSS
    \item Manage departmental sites (most running WordPress)
\end{itemize}}
\section*{Education}
\subsection*{UC Davis --- \textit{B.S. in Applied Physics}}
\subsubsection*{Sep 2012 - Jun 2016}
\end{minipage}%
\hspace*{0.5cm}
\begin{minipage}[t]{.2\textwidth}
\section*{Toolset}
\vspace{0.8em}
\begin{itemize}
    \item django
    \item python
    \item git
    \item linux
    \item alacritty
    \item vim
    \item tmux
    \item sway
    \item \LaTeX
\end{itemize}
\end{minipage}
\end{document}

\documentclass{article}

\usepackage{array}
\usepackage{enumitem}
\usepackage[letterpaper, portrait, top=0.5in, right=0.5in, left=0.5in, bottom=2.5in]{geometry}
\usepackage[colorlinks=true, linkcolor=linkColor, urlcolor=linkColor, citecolor=linkColor, anchorcolor=linkColor]{hyperref}
\usepackage{multirow}
\usepackage{titlesec}
\usepackage[usenames, dvipsnames]{xcolor}
\usepackage{fancyhdr}
\usepackage{fontspec}
\setmonofont{Hack Nerd Font}

\definecolor{blue}{HTML}{2079C7}
\definecolor{gray}{HTML}{666666}
\definecolor{linkColor}{HTML}{0000FF}

\newcolumntype{L}[1]{>{\raggedright\let\newline\\\arraybackslash\hspace{0pt}}m{#1}}
\newcolumntype{C}[1]{>{\centering\let\newline\\\arraybackslash\hspace{0pt}}m{#1}}
\newcolumntype{R}[1]{>{\raggedleft\let\newline\\\arraybackslash\hspace{0pt}}m{#1}}

\pagenumbering{gobble}

\renewcommand{\familydefault}{\sfdefault}
\renewcommand{\headrulewidth}{0.4pt}
\renewcommand{\footrulewidth}{0.4pt}

\setlength{\headheight}{69pt}
\setlist[itemize]{noitemsep, leftmargin=1em}

\titleformat{\section}
    {\color{blue}\normalfont\Large\bfseries}
    {\color{blue}\thesection}{1em}{\large\uppercase}
\titleformat{\subsection}
    {\normalfont\Large\bfseries}
    {\thesection}{1em}{\normalsize}
\titleformat{\subsubsection}
    {\color{gray}\normalfont\Large\bfseries}
    {\color{gray}\thesection}{1em}{\small}

\titlespacing*{\section}
    {0em}{0em}{0em}
\titlespacing*{\subsection}
    {0em}{1em}{0em}
\titlespacing*{\subsubsection}
    {0em}{0em}{0em}

\begin{document}

\noindent

\pagestyle{fancy}
\fancyhead[L]{
	\begin{tabular}{L{.4\textwidth} l l}
		\multirow{2}{*}{\Huge{\textbf{Alejandro Angulo} }} & \small{\texttt{}} \space \href{https://github.com/alejandro-angulo}{https://github.com/alejandro-angulo} & \small{\texttt{󰖟}} \space \href{https://alejandr0angul0.dev}{https://alejandr0angul0.dev} \\
		                                                   & {\small{\texttt{}} \space \href{mailto:\email}\email}                                                    & \small{\texttt{}} \space \phonenumber                                                    \\ % chktex 1
	\end{tabular}
}
\fancyfoot[L]{
	The source code for this document is available at \href{https://github.com/alejandro-angulo/resume}{https://github.com/alejandro-angulo/resume}\\
	If you have \href{https://nixos.wiki/wiki/Nix_package_manager}{nix} installed you can run the command below to generate the latest version\\
	\texttt{nix run --refresh --experimental-features 'nix-command flakes' github:alejandro-angulo/resume/main --  \textbackslash\\\ \ -e '\email' -p '\phonenumber'}  % chktex 8 chktex 32
}
\vspace*{-0.3in}
\hspace*{-2em}
\begin{minipage}[t]{.6\textwidth}
	\section*{Professional Experience}
	\subsection*{Sure  --- \textit{Staff Software Engineer}}
	\subsubsection*{February 2020--Present}
	\begin{itemize}
		\item Backend team lead on the service (Django app) responsible for
		      managing an insurance policy's lifecycle (quoting, binding, etc.)
		\item Working on transitioning toward a more data-driven design
		      (drastically reducing the time/complexity to onboard new clients)
		\item Help unstick teammates with various types of problems (failing test cases,
		      environment troubleshooting, suggestions on implementation, etc.)
		\item Investing in our developer experience (setting up tooling for
		      linting/formatting, investigating slow CI pipelines, improving
		      build
		      process, adding tooling to generate documentation, etc.)
		\item Participating in setting technical direction for teams across the
		      division
	\end{itemize}
	\subsection*{Everbridge (formerly NC4) --- \textit{Software Engineer}}
	\subsubsection*{October 2018--February 2020}
	\begin{itemize}
		\item Developed and maintained Python (Django app) and PHP applications
		      used to help notify clients of potential issues (fires, police
		      activity, etc)
		\item Migrated version control system from SVN to Git
		\item Re-architected Python application to improve modularity
		\item Introduced best practices (follow PEP8 for Python, begin linting code, etc.)
		\item Streamlined code review process (integrated with source control)
		\item Maintained and developed a RESTful API written in Python capable of generating
		      reports for end users
		\item Worked with product managers to implement ``behind the scenes'' changes
		      (upgrading language versions, databases, etc.)
	\end{itemize}
	\subsection*{MedQIA --- \textit{Software Engineer}}
	\subsubsection*{April 2017--September 2018}
	\begin{itemize}
		\item Developed and maintained a Java desktop application used by clinical readers
		      and lab technologists to assist in clinical trials
		\item Documented and validated systems to comply with FDA regulations
		\item Worked on Python applications that keep track of the status of scans
		\item Maintained and develop a web client written in PHP and JavaScript that provided
		      a dashboard for the data team
		\item Maintained and develop a RESTful API written in Python capable of generating
		      reports for end users
		\item Wrote Python scripts that plug in to a Java application in order to provide
		      additional functionality (dynamic loading of patients, automating certain tasks
		      for end users, etc.)
	\end{itemize}
	\subsection*{UC Davis College of Engineering IT Shared Services --- \textit{Student Web Developer and Applications Programmer}}
	\subsubsection*{May 2014--April 2017}
	\begin{itemize}
		\item Managed departmental websites (most running WordPress)
		\item Wrote PHP scripts to scrape data from third-party sites in order to visualize
		      energy consumption
		\item Helped ensure compliance with security standards
	\end{itemize}
\end{minipage}
\hspace{0.08\textwidth}
\begin{minipage}[t]{.3\textwidth}
	\section*{About Me}
	I'm a computer nerd. I've always enjoyed tinkering with computers and now I
	get to make a living from it. I'm excited about writing maintainable
	software and am looking to join a team that values having a good developer
	experience.

	Outside of my professional work, lately I've taken an interest in
	\href{https://nixos.org/}{nix}. I've been leveraging it to maintain my
	personal servers/machines (I self host a bunch of services in my private
	network). I've dabbled a bit in hardware as well and I built the keyboard
	I'm typing on right now. If a computer is involved, I'm probably interested
	in learning more!

	\vspace{1em}
	\section*{Related links}
	\begin{itemize}
		\item \href{https://github.com/alejandro-angulo/zmk-config}{keyboard configuration}
		\item \href{https://github.com/alejandro-angulo/dotfiles/}{nix configuration}
	\end{itemize}
	\section*{Toolset}
	Some tools I use regularly in my current role:
	\vspace{-0.6em}
	\begin{itemize}
		\item docker
		\item git
		\item linux
		\item tmux
		\item python
		\item neovim
	\end{itemize}
	\section*{Education}
	\subsection*{UC Davis}
	\subsubsection*{B.S. in Applied Physics}
	\subsubsection*{2012--2016}
\end{minipage}
\end{document}
